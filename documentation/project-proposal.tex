%%%%%%%%%%%%%%%%%%%%%%%%%%%%%%%%%%%%%%%%%%%%%%%%%%
%% TOP MATTER
%%%%%%%%%%%%%%%%%%%%%%%%%%%%%%%%%%%%%%%%%%%%%%%%%%

\documentclass{article}

% (1) Packages.
\usepackage{amsfonts}
\usepackage{amsmath}
\usepackage{amssymb}
\usepackage{amsthm}
\usepackage[yyyymmdd]{datetime}
\usepackage[left=2cm,
            right=2cm,
            top=2cm,
            bottom=2cm,
            headheight=18pt,
            includehead,
            includefoot,
            heightrounded]{geometry}
\usepackage{graphicx}
\usepackage{lipsum}
\usepackage{listings}
\usepackage{parskip}
\usepackage[table,xcdraw]{xcolor}

% (2) Commands.
\AtBeginDocument{\renewcommand{\hbar}{\hslash}}
\newcommand{\N}{\ensuremath{\mathbb{N}}}
\newcommand{\Z}{\ensuremath{\mathbb{Z}}}
\newcommand{\Q}{\ensuremath{\mathbb{Q}}}
\newcommand{\R}{\ensuremath{\mathbb{R}}}
\newcommand{\C}{\ensuremath{\mathbb{C}}}
\renewcommand{\dateseparator}{-}

% (3) Miscellaneous.
\graphicspath{ {../../Images/} }
\lstset{
  numbers = left,
  numberstyle = \small,
  numbersep = 8pt,
  frame = single,
  language = C++,
  framexleftmargin = 16pt
}
\title{
\textsc{Final Project Proposal} \\
A Simple Artificial Neural Network Written in C++
}
\author{
Liam Mulhall \\
Instructor: Shayon Gupta \\
TA: Alexander Curtiss \\
Data Structures
}
\date{\today}

%%%%%%%%%%%%%%%%%%%%%%%%%%%%%%%%%%%%%%%%%%%%%%%%%%
%% BODY
%%%%%%%%%%%%%%%%%%%%%%%%%%%%%%%%%%%%%%%%%%%%%%%%%%

\begin{document}

\maketitle

\section{General Idea}

For my final project, I intend to create a simple neural network in C++. I
don't mind if my program isn't terribly impressive, but I do want it to be
correct and efficient. My main goal is to understand the basics of neural
networks.

\section{Design}

The design of the neural network is fairly simple. We have a layer of input
neurons, a layer of hidden neurons, and a layer of output neurons. Any given
neuron has a weighted connection to every neuron in the next layer.

\begin{figure}[h]
  \centering
  \includegraphics[scale=0.1]{Neural-Network}
  \caption{A Simple 3-4-2 Neural Network}
\end{figure}

The input neurons will simply hold input values. The hidden neurons will
receive inputs, perform mathematical operations on those inputs, and then they
may send the modified inputs to neurons in the next layer, which is the output
layer in our case.

If we want the neural network to train itself, we have to use something called
backpropogation. In backpropogation, we compare the received output to the
desired output, and then we adjust the weight of the connections, which are
used in the mathematical operations that ultimately determine the output.

Each layer except for the output layer will have a bias neuron. Bias neurons
always send an output (typically \( 1 \)) to the neurons in the next layer.

\section{Underlying Data Structure}

The underlying data structure of our neural network will be an implicit
weighted graph. It will be weighted because each connection has a weight
associated with it.

\section{Problem}

The problem we hope to solve is the XOR problem. We hope to train our neural
network to behave like an XOR gate.

\begin{table}[h]
  \centering
    \begin{tabular}{|c|c|c|}
      \hline
      \multicolumn{2}{|c|}{\textbf{INPUT}} & \textbf{OUTPUT} \\ \hline
        A & B & A XOR B \\ \hline
        0 & 0 & 0 \\ \hline
        0 & 1 & 1 \\ \hline
        1 & 0 & 1 \\ \hline
        1 & 1 & 0 \\ \hline
    \end{tabular}
  \caption{XOR Truth Table}
\end{table}

\section{Features}

The ``Network'' class will have the following functions:
\begin{enumerate}
  \item
  A constructor.
  \item
  \verb!networkFeedForward! --- A function that is used to generate outputs.
  \item
  \verb!backPropogate! --- A function that compares desired output with
  received output and then adjusts weights accordingly.
  \item
  A variety of functions that print various results and statistics.
\end{enumerate}

The ``Neuron'' class will have the following functions:
\begin{enumerate}
  \item
  A constructor.
  \item
  Getter and setter functions for the output value.
  \item
  \verb!neuronFeedForward! --- A function that is used to generate outputs.
  \item
  A variety of mathematical functions for manipulating inputs.
\end{enumerate}

\section{Self-Imposed Deadlines}

This program can be broken up into three parts: (1) the ``Network'' class (2)
the ``Neuron'' class, and (3) working out the details of training the network,
i.e., making a training file, reading from said file, and making the output of
the program look nice.

Since I'll have time to work on this over fall break, I'll try to make the
``Optimistic Deadlines.''

\textbf{Optimistic Deadlines}

\begin{enumerate}
  \item
  2018-11-20
  \item
  2018-11-24
  \item
  2018-11-26
\end{enumerate}

\textbf{Realistic Deadlines}

\begin{enumerate}
  \item
  2018-11-26
  \item
  2018-12-02
  \item
  2018-11-08
\end{enumerate}

\end{document}
